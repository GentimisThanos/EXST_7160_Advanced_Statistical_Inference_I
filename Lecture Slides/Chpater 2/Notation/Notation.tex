\documentclass[12pt]{article}



%Basic Preamble:


\usepackage{xcolor}
\usepackage{mathtools}
\usepackage{booktabs}
\usepackage[english]{babel} % English language/hyphenation
\usepackage[protrusion=true,expansion=true]{microtype} % Better typography
\usepackage{amsmath,amsfonts,amsthm}
\usepackage{ amssymb }
\usepackage{graphicx}
\usepackage{array}
\usepackage[margin=1in,hmarginratio=1:1,top=20mm,columnsep=20pt]{geometry} % Document margins
\usepackage{booktabs} % Horizontal rules in tables
\usepackage{natbib}
\usepackage{setspace}
%\usepackage{subcaption}
\bibliographystyle{apalike}
%Additional Preamble
\usepackage{microtype} % Slightly tweak font spacing for aesthetics
\usepackage{subfig}
\usepackage{tabularx}
\usepackage[font = small,labelfont=bf,textfont=it]{caption} % Custom captions under/above floats in tables or figures
\linespread{1.05} % Line spacing
\usepackage{footnote}
\usepackage{algorithm}% http://ctan.org/pkg/algorithms
\usepackage[noend]{algpseudocode}% http://ctan.org/pkg/algorithmicx
\usepackage{enumitem}

\newtheorem{theorem}{Theorem}
\newtheorem{prop}{Proposition}
%Macros

\makeatletter
\newcommand{\distas}[1]{\mathbin{\overset{#1}{\kern\z@\sim}}}%
\newcommand{\bm}[1]{\mathbf{#1}}
\newsavebox{\mybox}\newsavebox{\mysim}
\newcommand{\distras}[1]{%

  \savebox{\mybox}{\hbox{\kern3pt$\scriptstyle#1$\kern3pt}}%

  \savebox{\mysim}{\hbox{$\sim$}}%

  \mathbin{\overset{#1}{\kern\z@\resizebox{\wd\mybox}{\ht\mysim}{$\sim$}}}%

}
\newcolumntype{C}[1]{>{\centering\let\newline\\\arraybackslash\hspace{0pt}}m{#1}}
\newcommand{\hilight}[1]{\colorbox{yellow}{#1}}
\setlength\heavyrulewidth{1.5pt} %thick top rule
\newcommand{\tr}[1]{\textrm{#1}}
\newcommand{\crd}[1]{{\color{red}{#1}}}
\newcommand{\be}{\begin{equation}}
\newcommand{\ee}{\end{equation}}
\newcommand{\bi}{\begin{itemize}}
\newcommand{\ei}{\end{itemize}}
\newcommand{\ben}{\begin{enumerate}}
\newcommand{\een}{\end{enumerate}}
\newcommand{\stb}{\State $\bullet$ \;}

\usepackage{amsmath}
\DeclareMathOperator*{\argmax}{arg\,max}
\DeclareMathOperator*{\argmin}{arg\,min}

\makeatother
%\numberwithin{equation}{section}
\let\oldbibliography\thebibliography
\renewcommand{\thebibliography}[1]{\oldbibliography{#1}
\setlength{\itemsep}{0pt}} %Reducing spacing in the bibliography.


\usepackage{makecell}
\usepackage{amsmath}
%----------------------------------------------------------------------------------------

%	TITLE SECTION

%----------------------------------------------------------------------------------------

\title{Some Note for OSS paper}
\date{}
\author{}
%----------------------------------------------------------------------------------------


\begin{document}

\begin{itemize}
    \item $Y$ or $X:$ a random variable
    \item $\mathbf{Y} = (Y_{1}, \cdots, Y_{n})$ or $\mathbf{X} = (X_{1}, \cdots, X_{n}):$ a random row vector
    \item $\mathbf{X}^{T}:$ transpose of $\mathbf{X},$ i.e., a random column vector.
    \item $y$  observed data from $Y$ ($x$ is observed data from $X$)
    \item $\mathbf{y} = (y_{1}, \cdots, y_{n})$ observed data from $\mathbf{Y}$ ($x = (x_{1}, \cdots, x_{n})$ is observed data from $\mathbf{X}$)
    \item $\boldsymbol{\theta} = (\theta_{1}, \cdots, \theta_{b}):$ a vector contains b unknown parameters.
    \item iid: independent and identical distributed
    \item $f(y; \boldsymbol{\theta})$ (or $f(x; \boldsymbol{\theta})$): density function of $Y$ (or $X$) with parameters $\boldsymbol{\theta}$ 
    \item $F(y; \boldsymbol{\theta})$ (or $F(x; \boldsymbol{\theta})$): distribution function of $Y$ (or $X$) with parameters $\boldsymbol{\theta}$
    \item $f(\mathbf{y};\boldsymbol{\theta})$  (or $F(\mathbf{y};\boldsymbol{\theta})$): joint density function of $\mathbf{Y}$  (or $\mathbf{X}$) with parameters $\boldsymbol{\theta}$:
    \item $F(\mathbf{y}; \boldsymbol{\theta})$  (or $F(\mathbf{x}; \boldsymbol{\theta})$): joint distribution function of $\mathbf{Y}$  (or $\mathbf{X}$) with parameters $\boldsymbol{\theta}$
    \item $L(\boldsymbol{\theta}|\mathbf{y}):$ the likelihood function of $\boldsymbol{\theta}$ given data $\mathbf{y}$.
    \item $\hat{\boldsymbol{\theta}}_{MLE}:$ the maximum likelihood estimator (MLE) of $\boldsymbol{\theta}$.
    \item $\argmax_{\boldsymbol{\theta}} g(\boldsymbol{\theta}):$ The value of $\boldsymbol{\theta}$ that makes $g(\boldsymbol{\theta})$ achieve maximum.
    \item $\frac{\partial g(\boldsymbol{\theta})}{\partial \boldsymbol{\theta}}:$ the derivative of function $g$ with respect to $\boldsymbol{\theta}$
    \item $\bar{y}:$ the sample average, i.e., $\frac{\sum^{n}_{i=1}y_{i}}{n}$
    \item $\prod^{n}_{i=1}x_{i}$: The cumulative product from $x_{1}$ to $x_{n}$
    \item $g^{-1}:$ the inverse of $g(\cdot)$.
\end{itemize}
\end{document}
